%Florian Bogner & Alexander Palmrich

\documentclass[a4paper,12pt]{paper}
\usepackage{fullpage}
\usepackage[utf8]{inputenc}
%\usepackage[ngerman]{babel}
\usepackage[english]{babel}
\usepackage{amsmath}
\usepackage{amssymb}
\usepackage{latexsym}
\usepackage{mathtools}
\usepackage{listings}
\usepackage{algorithm}
\usepackage{algpseudocode}
\usepackage{graphicx}
\usepackage{booktabs}
\usepackage{hhline}
\usepackage{amsthm}

\theoremstyle{plain}
\newtheorem{thm}{Theorem}[section] % reset theorem numbering for each chapter

\theoremstyle{definition}
\newtheorem{defn}[thm]{Definition} % definition numbers are dependent on theorem numbers
\newtheorem{exmp}[thm]{Example} % same for example numbers
\newtheorem{lem}[thm]{Lemma}
%https://tex.stackexchange.com/questions/45817/theorem-definition-lemma-problem-numbering


\begin{document}

\begin{titlepage}
\huge
\centering
Border Propagation: a novel approach to determining Slope Region Decompositions

\vfill

\normalsize
Florian Bogner \& Alexander Palmrich
\end{titlepage}




\tableofcontents
\newpage

%% high level abstract: %%
% - motivate slope region with 2D function surface
% - define slope region for any dimension
% - reference known algorithm
% - motivate new algorithm for continuous case
% - describe discrete algorithm
% - many many fancy pictures

\section{Abstract}

TODO

\section{Definitions}

In this and the following chapters we will unless otherwise specified consider a closed $\Omega \subset \mathbb{R}^n$ with $n > 1$. We consider a continuous function $f: \Omega \mapsto \mathbb{R}$, which may be interpreted as a gray-scale image.

\begin{defn}
A path $\gamma: [a,b] \mapsto \Omega$ is called \emph{monotonic} if and only if it is continuous and
\begin{align*}
\forall & s, t \in [a,b]: s < t \Rightarrow f(\gamma(s)) \leq f(\gamma(t)) ~ \lor \\
\forall & s, t \in [a,b]: s < t \Rightarrow f(\gamma(s)) \geq f(\gamma(t))
\end{align*}
\end{defn}

\begin{defn}
Let $R \subset \Omega$. $R$ is called \emph{slope region} or \emph{monotonically connected} if and only if for all $x, y \in R$ there exists a monotonic path $\gamma: [a,b] \mapsto R$ with $\gamma(a) = x$ and $\gamma(b) = y$.
\end{defn}

\begin{defn}
A family of sets $\left\{ A_i \subset \Omega \mid i \in I \right\}$ is called a \emph{slope region decomposition} if and only if:
\begin{itemize}
\item $A_i$ is a slope region for all $i \in I$
\item $\forall i,j \in I: i \neq j \Rightarrow A_i \cap A_j = \emptyset$.
\item $\bigcup_{i\in I} A_i = \Omega$
\end{itemize}
\end{defn}

%\begin{defn}
%A slope region decomposition $\left\{ R_i \subset \Omega \mid i \in I \right\}$ is called \emph{maximally coarse} or simply \emph{coarse} if and only if there is no $J \subset I$ with $|J| \geq 2$ and
%\begin{equation*}
%\left\{ R_i \subset \Omega \mid i \in I \setminus J \right\} \cup \left\{ \bigcup_{j\in J} R_j \right\}
%\end{equation*}
%\end{defn}


\begin{defn}
Consider two slope region decompositions $\mathcal{A} = \left\{ A_i \subset \Omega \mid i \in I \right\}$ and $\mathcal{B} = \left\{ B_j \subset \Omega \mid j \in J \right\}$. We call $\mathcal{A}$ \emph{coarser than} $\mathcal{B}$, in Symbols $\mathcal{A} \succeq \mathcal{B}$ if and only if
\begin{equation*}
\forall j \in J ~ \exists i \in I: B_j \subset A_i.
\end{equation*}
\end{defn}

\begin{lem}
$\succeq$ is a partial order, i.e. fulfills reflexivity, antisymmetry and transitivity.

\emph{Proof:} trivial \hfill $\Box$
\end{lem}

\begin{defn}
A slope region decomposition $\mathcal{A}$ is called \emph{maximally coarse} or simply \emph{coarse} if and only if there is no other coarser slope region decomposition.
\end{defn}

\end{document}
































