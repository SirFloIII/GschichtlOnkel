%Florian Bogner & Alexander Palmrich

\documentclass[a4paper,12pt]{paper}
\usepackage{fullpage}
\usepackage[utf8]{inputenc}
%\usepackage[ngerman]{babel}
\usepackage[english]{babel}
\usepackage{amsmath}
\usepackage{amssymb}
\usepackage{latexsym}
\usepackage{mathtools}
\usepackage{listings}
%\usepackage{algorithm}
%\usepackage{algpseudocode}
\usepackage{graphicx}
\usepackage{booktabs}
\usepackage{hhline}
\usepackage{amsthm}

\theoremstyle{plain}
\newtheorem{thm}{Theorem}[section] % reset theorem numbering for each chapter

\theoremstyle{definition}
\newtheorem{defn}[thm]{Definition} % definition numbers are dependent on theorem numbers
\newtheorem{exmp}[thm]{Example} % same for example numbers
%https://tex.stackexchange.com/questions/45817/theorem-definition-lemma-problem-numbering


\begin{document}

\begin{titlepage}
\huge
\centering
Border Propagation: A Novel Approach To Determining Slope Region Decompositions

\vfill

\normalsize
Florian Bogner \& Alexander Palmrich
\end{titlepage}




\tableofcontents
\newpage

%% high level abstract: %%
% - motivate slope region with 2D function surface
% - define slope region for any dimension
% - reference known algorithm
% - motivate new algorithm for continuous case
% - describe discrete algorithm
% - many many fancy pictures

\section{Abstract}

TODO

\section{Motivating Slope Regions}

In this section we try to develop an intuitive understanding of the term \emph{slope region} and its generalization to dimensions higher than 2. The concise definitions of the terms already employed here is reserved for the next section.

Consider an image, either gray-scale or in color. If it is a color image, it can be decomposed into its color channels, which can individually be read as gray-scale images. We now think of the light intensity (i.e. the image value) of one such gray-scale image as the height of a landscape, yielding a surface in 3D space. The surface will have hills in areas where the image is bright, and will have dales in dark areas. Please refer to figure TODO REF for a visualization!

Our aim now is to partition the surface into \emph{regions} (i.e. subsets) in a particular way: We require each region to consist only of a single slope, by which we mean that we can ascend (or descend) from any given point of the region, to any other given point of the region, along a path that runs entirely within the region. Such a decomposition is not unique, but we can at least try to get a partition \emph{as coarse as possible}, meaning that we merge slope regions if the resulting subset is still a slope region, and we iterate this until it can be done no more. There might be many different such coarsest slope decompositions, but we accept any one of them.

The criterion we used to describe slopes, any two points being connected by either an ascending or a descending path, can easily be used in higher dimensions. Think of a computer tomography scan, which will yield gray-scale data, but not just as a 2D image, but rather on a 3D volume.
Now, if you are inclined to visualize a fourth spacial dimension, you can think of that fourth direction as the height of a 3D hyper-surface in 4D space, with the height encoded in the tomography scan. But imagining this is difficult and not required, because our slope region criterion works just fine here: We want to partition the 3D volume, such that any two points in a region can be connected via an either ascending or descending path within the region. Recall that \emph{ascending} and \emph{descending} refers to the brightness value of the tomography scan as we move in the volume.

By abstracting from image and tomography to a real function defined on some subset of $\mathbb{R}^n$ (think of it as the brightness function!), and by rigorously defining a coarsest slope decomposition, we can lift the concept to arbitrary dimensions in a mathematically concise fashion.

\section{Definitions}

In this and the following chapters we will unless otherwise specified consider a closed $\Omega \subset \mathbb{R}^n$ with $n > 1$. We consider a continuous function $f: \Omega \mapsto \mathbb{R}$, which may be interpreted as a gray-scale image.

\begin{defn}
A path $\gamma: [a,b] \mapsto \Omega$ is called monotonic, 
\end{defn}



\section{Motivating The New Algorithm}
We will now work our way to the central insights on which the border propagation algorithm hinges.

Slope regions can be constructed and grown in a straight-forward iterative manner by sweeping through the function values from lowest to highest. This is similar to the intuition employed in Morse theory. TODO cite morse theory. Visualize a compact 2D surface in 3D space. Initially, our decomposition is empty, i.e. there are no slope regions (and we thus don't yet have an actual \emph{decomposition}).

Starting at a global minimum, we add a new region, containing only the argmin (i.e. a single point on the 2D image where the minimal value is taken). Imagine a water level rising from below the surface, up to the point of first contact. Now, there might be many points where the global minimum is taken. This will either be due to flat regions (\emph{plateaus}) on the surface, which we want to include into the single existing region, or it will be due to individual dales, which all have their lowest point at the same height.

In this second case, we can't put the points into the existing region, because we would not be able to get from one argmin to another via a monotonic path. Instead, we need to add a new region for each individual dale, that is to say we need to look at the connected components of the levelset and assign to each component a new region.

As the water rises, we can add points to an existing region growing it upwards if they are just outside the region\footnote{Why can we do that? By adding only points which are connected to the region we ensure path-connectedness, and by growing the region upwards, we can construct ascending paths from old points to new ones. TODO where do we get connectedness of increment from?}. Otherwise they correspond to a local minimum and have to be dealt with as before, by opening a new region for each point (or rather, each connected component).

With the water rising further still, the regions will grow up to a point where they meet. Any such point is a saddle point, and we have to account for it the next time we want to grow any one of the touching regions. The saddle point connects the edges of the regions which meet in it. Those edges are at the height of the water level.
\end{document}
































