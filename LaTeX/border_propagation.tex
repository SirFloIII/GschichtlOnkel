%Florian Bogner & Alexander Palmrich

\documentclass[a4paper,12pt]{paper}
\usepackage{fullpage}
\usepackage[utf8]{inputenc}
%\usepackage[ngerman]{babel}
\usepackage[english]{babel}
\usepackage{amsmath}
\usepackage{amssymb}
\usepackage{latexsym}
\usepackage{mathtools}
\usepackage{listings}
\usepackage{algorithm}
\usepackage{algpseudocode}
\usepackage{graphicx}
\usepackage{booktabs}
\usepackage{hhline}
\usepackage{amsthm}

\theoremstyle{plain}
\newtheorem{thm}{Theorem}[section] % reset theorem numbering for each chapter

\theoremstyle{definition}
\newtheorem{defn}[thm]{Definition} % definition numbers are dependent on theorem numbers
\newtheorem{exmp}[thm]{Example} % same for example numbers
%https://tex.stackexchange.com/questions/45817/theorem-definition-lemma-problem-numbering


\begin{document}

\begin{titlepage}
\huge
\centering
Border Propagation: a novel approach to determining Slope Region Decompositions

\vfill

\normalsize
Florian Bogner \& Alexander Palmrich
\end{titlepage}




\tableofcontents
\newpage

%% high level abstract: %%
% - motivate slope region with 2D function surface
% - define slope region for any dimension
% - reference known algorithm
% - motivate new algorithm for continuous case
% - describe discrete algorithm
% - many many fancy pictures

\section{Abstract}

TODO

\section{Definitions}

In this and the following chapters we will unless otherwise specified consider a closed $\Omega \subset \mathbb{R}^n$ with $n > 1$. We consider a continuous function $f: \Omega \mapsto \mathbb{R}$, which may be interpreted as a gray-scale image.

\begin{defn}
A path $\gamma: [a,b] \mapsto \Omega$ is called monotonic, 
\end{defn}

\end{document}
































